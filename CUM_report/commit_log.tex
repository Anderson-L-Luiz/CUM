\documentclass{article}
\usepackage[utf8]{inputenc}
\usepackage{parskip}
\usepackage{hyperref}
\usepackage[T1]{fontenc}
\usepackage{lmodern}
\usepackage{amsmath}
\usepackage{xcolor}      % For colored text
\usepackage{courier}      % For pcr font family
\usepackage{xurl}         % For better URL handling
\usepackage{minted}       % For syntax highlighting (optional)
\usepackage{geometry}     % For page layout
\geometry{a4paper, margin=1in}

\title{Commit Summary Report (CUM Report)}
\author{Generated by Git Post-Commit Hook}
\date{\today}

\begin{document}
\maketitle
\begin{abstract}
This document contains summaries of commits, generated automatically after each commit. Each commit is a section, and each modified file within that commit is presented as a subsection with its AI-generated summary. Commit details are provided at the end of each section. If a file's diff is large, its summary might be generated from multiple parts.
\end{abstract}
\tableofcontents
\newpage

\section{Introduction}
This document provides a log of commit summaries. Each main section corresponds to a single Git commit. Within each commit section, individual subsections detail the AI-generated summary for each modified file. Key details about the commit (hash, author, date) are listed at the end of each section's content.
\section{Commit 5aee97c by Anderson-L-Luiz (2025-05-20 06:46:27 +0000)}
\subsection{File: \texttt{CUM\_report/commit\_log.tex}}\n{\fontfamily{pcr}\selectfont\n  The code changes in the commit log file are related to the `install\_cum\_coder.sh` script, which is used to install the `cum----coder.sh` hook script. The script has been modified to use the `/v1/completions` endpoint instead of `/v1/chat/completions` to handle the model's name and formatting of prompts. Additionally, the maximum length of the diff file has been increased to 100,000 characters to accommodate longer code snippets. The changes also include some improvements to the error handling and logging. Overall, these changes are likely to make the script more efficient and effective in generating summaries for code changes.\n}\n\n\subsection{File: \texttt{install\_cum\_coder.sh}}\n{\fontfamily{pcr}\selectfont\n\{\textbackslash\{\}bfseries Summary for Diff Part 1 of 2:\}\textbackslash\{\}par\textbackslash\{\}n  \# Close out the Instruct block for this chunk
+                else
+                    \# Response parsing for /v1/completions
+                    if echo "\$api\_response\_file\_chunk" | jq --e '.choices[0].text' \textgreater{} /dev/null 2\textgreater{}\&1; then
+                        extracted\_text\_file\_chunk=\$(echo "\$api\_response\_file\_chunk" | jq --r '.choices[0].text')
+                        summary\_text\_for\_this\_chunk="\$extracted\_text\_file\_chunk"
+                        echo --e "   ✅ Summary for \$file\_path (Part \$chunk\_num/\$num\_chunks) received."
                    else
                        echo "   ❌ Hook Warning: Failed to extract summary text for \$file\_path (Part \$chunk\_num/\$num\_chunks) from /v1/completions response or response was malformed." \textgreater{}\&2
                        echo "   Raw API Response for \$file\_path (Part \$chunk\_num/\$num\_chunks): \$api\_response\_file\_chunk" \textgreater{}\&2 \# Log raw response for debugging
                        summary\_text\_for\_this\_chunk="Automated summary generation failed for this part: Could not parse LLM response or response was malformed from /v1/completions."
                    fi
                 fi
             fi
         fi
 
         if [ --z "\$summary\_text\_for\_file" ]; then
             echo "   ❌ Hook Error: No summary text was generated for \$file\_path. Skipping output." \textgreater{}\&2
+        else
+            final\_summary\_for\_file\_parts+=\$summary\_text\_for\_this\_chunk
         fi
     fi
 
     \# Add to LaTeX output
--    file\_summary\_text\_escaped\_for\_latex+="\$(echo --e "\$summary\_text\_for\_file" | sed 's/\textbackslash\{\}\textbackslash\{\}/\textbackslash\{\}\textbackslash\{\}textbackslash/g' | sed 's/\&/\textbackslash\{\}\textbackslash\{\}\&/g' | sed 's/\%/\textbackslash\{\}\textbackslash\{\}\%/g' | sed 's/\_/\textbackslash\{\}\textbackslash\{\}\_/g' | sed 's/\^{}/\textbackslash\{\}\textbackslash\{\}\textbackslash\{\}\textbackslash\{\}begin\{quotation\}\textbackslash\{\}\textbackslash\{\}small\textbackslash\{\}\textbackslash\{\}begin\{verbatim\}/' | sed 's/\$/\textbackslash\{\}\textbackslash\{\}end\{verbatim\}\textbackslash\{\}\textbackslash\{\}end\{quotation\}/')"
--    echo "\$file\_summary\_text\_escaped\_for\_latex" \textgreater{}\textgreater{} "\$TEMP\_SUMMARY\_FILE\_PATH"
     echo "   ℹ️ Summary for \$file\_path added to LaTeX file."
 done
@@ --299,7 +387,7 @@ fi
 
 \# Write LaTeX header
 echo "\textbackslash\{\}\textbackslash\{\}documentclass\{article\}
--\textbackslash\{\}\textbackslash\{\}usepackage\{xcolor\}      \% For colored text
--\textbackslash\{\}\textbackslash\{\}usepackage\{courier\}      \% For pcr font family
--\textbackslash\{\}\textbackslash\{\}usepackage\{xurl\}         \% For better URL handling
+\textbackslash\{\}\textbackslash\{\}usepackage\{xcolor\}      \% For colored text
+\textbackslash\{\}\textbackslash\{\}usepackage\{courier\}      \% For pcr font family
+\textbackslash\{\}\textbackslash\{\}usepackage\{xurl\}         \% For better URL handling
+\textbackslash\{\}\textbackslash\{\}usepackage\{minted\}       \% For syntax highlighting (optional)
+\textbackslash\{\}\textbackslash\{\}usepackage\{geometry\}     \% For page layout
+\textbackslash\{\}\textbackslash\{\}geometry\{a4paper, margin=1in\}
 
 \textbackslash\{\}\textbackslash\{\}title\{Commit Summary Report (CUM Report)\}
 \textbackslash\{\}\textbackslash\{\}author\{Generated by Git Post--Commit Hook\}
@@ --308,7 +397,7 @@ echo "\textbackslash\{\}\textbackslash\{\}maketitle
 \textbackslash\{\}\textbackslash\{\}begin\{abstract\}
 This document contains summaries of commits, generated automatically after each commit. Each commit is a section, and each modified file within that commit is presented as a subsection with its AI--generated summary. Commit details are provided at the end of each section. If a file's diff is large, its summary might be generated from multiple parts.
--\textbackslash\{\}\textbackslash\{\}end\{abstract\}
 \textbackslash\{\}\textbackslash\{\}tableofcontents
 \textbackslash\{\}\textbackslash\{\}newpage
+\textbackslash\{\}\textbackslash\{\}end\{abstract\}
+\textbackslash\{\}\textbackslash\{\}tableofcontents
+\textbackslash\{\}\textbackslash\{\}newpage
 
 \% Summary for each file
 \textbackslash\{\}\textbackslash\{\}input\{\$TEMP\_SUMMARY\_FILE\_PATH\}" \textgreater{} "\$TEMP\_LATEX\_FILE\_PATH"
@@ --321,24 +412,17 @@ if [ \$? --eq 0 ]; then
 
 \# Use LaTeX to generate PDF
 echo "  ℹ️ Generating LaTeX PDF file..."
--pdflatex "\$TE\textbackslash\{\}n\textbackslash\{\}n\{\textbackslash\{\}bfseries Summary for Diff Part 2 of 2:\}\textbackslash\{\}par\textbackslash\{\}n  This is the final part of the diff for the commit 5aee97c. The changes are likely intended to improve the readability and formatting of the output from the vLLM API. The code makes use of several new variables, including VLLM\_API\_URL and MODEL\_NAME, which are likely intended to be set by the user. The changes also include additional error handling and debugging information. Overall, these changes are likely intended to improve the usability and functionality of the script.\textbackslash\{\}n\textbackslash\{\}n\n}\n\n
{\color{blue}\small % Start color blue and make text small\nCommit: \texttt{5aee97cc8e0a43cba1e0cb73b19f82c18bcfdf7d} \\\nAuthor: \texttt{Anderson-L-Luiz}\\\nDate: \texttt{2025-05-20 06:46:27 +0000}\n} % End color blue\n
\hrulefill

\section{Commit b3122e4 by Anderson-L-Luiz (2025-05-27 11:37:53 +0000)}
\subsection{File: \texttt{README.md}}\n{\fontfamily{pcr}\selectfont\n  This is a README file for a software project called C.U.M. (Commit User Modification). The changes in this commit add a new section to the file, which includes a link to a script called "install\_cum\_coder.sh". The new section is likely intended to provide instructions on how to install and use the software. The changes also update the description of the software to include a new sentence about the software being an "experience" rather than just a tool. Overall, these changes likely make it easier for users to install and use the software, and provide a better overall description of what C.U.M. does.\n}\n\n\subsection{File: \texttt{install\_cum\_coder.sh}}\n{\fontfamily{pcr}\selectfont\n  This script is a `git` hook that runs after a commit is made. It is designed to summarize the code changes made in the commit, and ask the user to explain what the changes likely achieve at a high level and their potential impact or purpose.

The script first constructs a prompt that includes the file path and commit hash, and then checks if the commit contains any changes to the repository. If there are changes, the script divides the changes into smaller chunks and prompts the user to summarize each chunk. The user is asked to explain what the changes likely achieve at a high level and their potential impact or purpose.

The script also checks if the hook is configured correctly and provides instructions for the user to fix any issues.

Overall, this script helps developers to quickly understand the impact of their code changes and to document the changes in a structured way.\n}\n\n
{\color{blue}\small % Start color blue and make text small\nCommit: \texttt{b3122e4d01295a9011a01972be45f002ab776fb1} \\\nAuthor: \texttt{Anderson-L-Luiz}\\\nDate: \texttt{2025-05-27 11:37:53 +0000}\n} % End color blue\n
\hrulefill

\end{document}
