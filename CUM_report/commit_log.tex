\documentclass{article}
\usepackage[utf8]{inputenc}
\usepackage{parskip}
\usepackage{hyperref}
\usepackage[T1]{fontenc}
\usepackage{lmodern}
\usepackage{amsmath}
\usepackage{courier} % For pcr font family

\title{Commit Summary Report (CUM Report)}
\author{Generated by Git Post-Commit Hook}
\date{\today}

\begin{document}
\maketitle
\begin{abstract}
This document contains summaries of commits, generated automatically after each commit and included herein.
\end{abstract}
\tableofcontents
\newpage

\section{Introduction}
This document provides a log of commit summaries. Each section corresponds to a summary generated at the time of a commit.
The summaries are generated by an AI model based on the diff of the commit.

\section{Commit a5350b3 by Anderson-L-Luiz (2025-05-12 05:41:49 +0000)}
\subsection{AI Generated Summary}
{\fontfamily{pcr}\selectfont
 The commit changes from `a5350b3` modified the `commit\_log.tex` file to include a summary of the latest commit in a Git repository. This summary includes the date and author of the commit and a brief summary of the changes made. The changes appear to have been made to improve the functionality and usefulness of the file. They likely have a high impact, as they automate the process of creating a summary of each commit and ensure that this summary is included in the file. This could potentially contribute to better record--keeping and collaboration within the team. The changes aim to create a more efficient and automated way of managing a commit log.

Specifically, the modifications in the commit made to the `install\_cum.sh` script involve the addition of the new git repo and hook installation process that creates a pre--push hook to generate a commit summary and include it in the LaTeX report. This change likely enhances the script's operation by enabling it to generate a summary of the latest commit and add it to the LaTeX report. The pull--requests (PR) 2 and PR 3 changes made to the updater file provide a new LaTeX and report generation utility, the newpdfpic\_CC\_AddHeader utility, and the update\_latex utility to convert the LaTeX to PDF. 

The changes made to the `INSTALL` script -- specifically the `install\_cum.sh` script -- increase its functionality. The addition of the new function generates on--the--fly a LaTeX report based on each commit. Furthermore, the changes ensure that the README.md and the github\_userguide include the commit log. It is to be noted that these changes are quite reasonably made soon after unreleased git versions get updated into the development branch.
}
\subsection{Commit Details}
\begin{itemize}
    \item \textbf{Commit Hash:} a5350b3dc72047d8f5c92e9b57bf659ab16ea15d
    \item \textbf{Author:} Anderson-L-Luiz
    \item \textbf{Date:} 2025-05-12 05:41:49 +0000
\end{itemize}
\hrulefill

\section{Commit 184a5cc by Anderson-L-Luiz (2025-05-12 05:44:41 +0000)}
\subsection{AI Generated Summary}
{\fontfamily{pcr}\selectfont
 The commit 184a5cc is related to a script that installs a software named "CUM." In this commit, there are some code changes in the script that may be optimizing or improving the performance of the script.

1. The script was previously limiting the output of diff to 5,000 characters, and now it is increased to 10,000 characters.
2. The output is trimmed to the new limit after sanitizing it using tr command.
3. The source of the change is difficult to derive from the code itself, but it likely has to do with improving the stability or efficiency of the script when dealing with very large diffs. The change could be made to ensure that larger diffing operations don't cause the script to run slowly or hang.

The impact of this change could be: improved stability, faster operation times and decreased memory usage when processing large changes. The purpose of the change could be to enable the script to handle larger diff outputs without any deterioration in performance.
}
\subsection{Commit Details}
\begin{itemize}
    \item \textbf{Commit Hash:} 184a5ccff7bacd96e6c9e6efc6091ea0d991071c
    \item \textbf{Author:} Anderson-L-Luiz
    \item \textbf{Date:} 2025-05-12 05:44:41 +0000
\end{itemize}
\hrulefill

\section{Commit f8c62d2 by Anderson-L-Luiz (2025-05-12 05:48:57 +0000)}
\subsection{AI Generated Summary}
{\fontfamily{pcr}\selectfont
 The given GitHub commit (f8c62d2) contains several changes in a Bash script named `install\_cum\_bkp1.sh`. This script is likely used as a post--commit Git hook that installs packages and sets up the Cum Report generation system for Git repositories. Here is a high--level summary of the changes:

1. `new file mode 100644`: A new file with mode 100644 (world--readable) is created, pointing to `/dev/null`, possibly to store hook--related information.

2. Hanging `set --e --o pipefail`: Ensures that the script won't proceed further if any command in the pipeline exits with a non--zero status. It avoids potential issues during script execution.

3. Checks for existence of Git repository directory, function for installing packages: The script first checks if it's running inside a Git repository and then finds the root directory of the repository. Then, a function named `install\_package` is created to install soft packages.

4. Creation of the CUM\_REPORT\_DIR and ensuring the necessary directory: The script then creates the directory `CUM\_report` (might be an existing directory as hinted by `checkpoint 4`). If not created, it prints an error message and exits. Afterward, it introduces a new directory `COMMIT\_LOG\_FILE` in the `CUM\_report` directory for the initial LaTeX summary.

5. Checks for existence of the initial LaTeX file, `commit\_log.tex`. If it does not exist, the script creates it.

6. Defines and installs the post--commit Git hook: The script adds the installation steps for the post--commit Git hook that runs after every commit. The hook updates the `pre--commit` hook to call `install\_cum\_bkp1.sh` first. The hook is also named `post--commit`.

7. Deletes old `pre--push` hook, if it exists \& creates `post--commit` directory: The script deletes any old `pre--push` hook before installing the new `post--commit` hook.

8. Creates the `post--commit` directory at the appropriate location.

The purpose of these changes are to set up a process of running certain scripts during or after a commit has taken place. This could include generating reports or performing actions based on the commit's content. 

Note: The script is currently configured for the directory structure and assumed file names to accommodate for Cum Report. Your actual directory structure in your repository may need to be adjusted accordingly.
}
\subsection{Commit Details}
\begin{itemize}
    \item \textbf{Commit Hash:} f8c62d273861408ce0dd29c74c21417f8b371c64
    \item \textbf{Author:} Anderson-L-Luiz
    \item \textbf{Date:} 2025-05-12 05:48:57 +0000
\end{itemize}
\hrulefill

\section{Commit 0005288 by Anderson-L-Luiz (2025-05-12 05:50:55 +0000)}
\subsection{AI Generated Summary}
{\fontfamily{pcr}\selectfont
 The commit 0005288 made several changes to the "lorem.md" file. These changes included adding new content, updating existing content, and modifying the indentation and formatting of the file.

One potential purpose of these changes could be to improve the readability and formatting of the file. The updated content could also include new information and insights that were previously missing or not properly conveyed. Additionally, the changes to the indentation and formatting could make it easier to navigate and understand the content of the file.

Ultimately, the impact and purpose of these changes will depend on the specific content and purpose of the "lorem.md" file, as well as the intended audience for the file.
}
\subsection{Commit Details}
\begin{itemize}
    \item \textbf{Commit Hash:} 0005288203b0ea206911f1ec7ad96eddd1c27e9b
    \item \textbf{Author:} Anderson-L-Luiz
    \item \textbf{Date:} 2025-05-12 05:50:55 +0000
\end{itemize}
\hrulefill

\section{Commit 5382f30 by Anderson-L-Luiz (2025-05-14 16:27:32 +0000)}
\subsection{AI Generated Summary}
{\fontfamily{pcr}\selectfont
 The commit 5382f30 made several changes to the `lorem.md` file, including updating the content, adding new content, modifying the formatting, and changing the indentation.

The potential purpose of these changes could be to improve readability, formatting, and content of the document. These changes might include updates to grammar, spelling, or information presented in the document. The changes to indentation and formatting make it easier to navigate and understand the content of the file.

The impact and purpose of these changes will depend on the specific content and purpose of the "lorem.md" file, as well as the intended audience for the document. The document may have been modified to correct errors, improve clarity, or add new information or insights to the text. Without more context about the purpose behind these changes, it's hard to say what the specific impact or purpose would be.
}
\subsection{Commit Details}
\begin{itemize}
    \item \textbf{Commit Hash:} 5382f308f1ec0c53e2ac12e2eb89bff5a59a923c
    \item \textbf{Author:} Anderson-L-Luiz
    \item \textbf{Date:} 2025-05-14 16:27:32 +0000
\end{itemize}
\hrulefill

\section{Commit c018eb8 by Anderson-L-Luiz (2025-05-15 07:59:22 +0000)}
\subsection{File: \texttt{CUM\_report/commit\_log.tex}}\n{\fontfamily{pcr}\selectfont\n  The changes made to the 'CUM\_report/commit\_log.tex' file likely aim to automate the process of generating a commit summary report for a Git repository. The changes include:

1. Creation of a script to install the necessary packages and generate a LaTeX file for the commit summary report.
2. Installation and setup of the CUM Report generation tool.
3. Deletion of the file "lorem.txt", which was likely used for testing or experimentation purposes.

These changes have the potential impact of allowing developers to easily generate a commit summary report for their Git repository after each commit. The report will include a summary of each commit, as well as a detailed list of each modified file and its AI--generated summary. The report will be generated using a LaTeX template and can be easily integrated into the Git workflow using a post--commit hook.\n}\n\n\subsection{File: \texttt{install\_cum\_alternative.sh}}\n{\fontfamily{pcr}\selectfont\n  The script 'install\_cum\_alternative.sh' was added in commit c018eb8, and it is intended to be a post--commit hook script that automates the installation of the CUM Report tool. The script checks for the presence of required packages (jq and curl), installs them if necessary, and then creates a directory for the CUM Report tool (if it doesn't already exist) and generates an initial LaTeX file (commit\_log.tex) in that directory.

The script sets up the environment to use bash, enables strict mode, and uses the 'set --e' and 'set --o pipefail' options to exit immediately if a command fails or if a pipeline fails, respectively.

The script then checks if it is running inside a Git repository and gets the root directory of the repository. It then installs the required packages (jq and curl) and creates the CUM Report directory if it doesn't exist. Finally, it generates an initial LaTeX file (commit\_log.tex) in the CUM Report directory.

The changes in this script likely achieve the following:

* Automate the installation of the CUM Report tool, including installing required packages and creating a directory for the tool
* Generate an initial LaTeX file (commit\_log.tex) in the CUM Report directory
* Set up the environment to use bash and enable strict mode

The potential impact of these changes is to make it easier to use the CUM Report tool with Git repositories, as the installation and setup process is automated.\n}\n\n\subsection{File: \texttt{install\_cum\_coder.sh}}\n{\fontfamily{pcr}\selectfont\n  This is a shell script that installs and sets up the "CUM Report" package, which is a tool for generating a summary of Git commits. The script is a post--commit hook, meaning it is executed automatically after each Git commit.

The script first checks if it is running inside a Git repository, and if not, it exits with an error. It then determines the root directory of the repository and installs two packages: `jq` and `curl`. The `jq` package is used for parsing JSON data, and the `curl` package is used for making HTTP requests.

Next, the script creates a directory called `CUM\_report` in the repository root, and creates an initial LaTeX file called `commit\_log.tex`. This file serves as a template for the CUM Report, and is used to generate the final report.

Finally, the script installs the `cum--report` package, which is the main package for generating the CUM Report. The package is installed using `npm`, and the script waits for the installation to complete before proceeding.

Overall, this script is a convenient way to automatically generate a summary of Git commits, and can be used to track changes to a project over time.\n}\n\n\subsection{File: \texttt{lorem.txt}}\n{\fontfamily{pcr}\selectfont\n  The changes to the 'lorem.txt' file likely involve removing or modifying the content of the file, which could impact the appearance or functionality of the application or website that uses the file. The deleted content appears to be a block of text, which could have been used for copywriting or other purposes. The modifications to the file may have been made to improve the readability or accessibility of the content, or to update the text to be more relevant or accurate. The changes may also have been made to improve the overall user experience or to enhance the appearance of the application or website.\n}\n\n
{\color{blue}\small % Start color blue and make text small\nCommit: \texttt{c018eb809cd8f16a63a283795a6fd023d13605c3} \\\nAuthor: Anderson-L-Luiz \\\nDate: 2025-05-15 07:59:22 +0000\n} % End color blue\n
\hrulefill

\end{document}
